\batchmode %% Suppresses most terminal output.
\documentclass{article}
\setlength{\textwidth}{360pt}
\setlength{\textheight}{541pt}
\usepackage{latexsym}
\usepackage{ifthen}
% \usepackage{color}
%%%%%%%%%%%%%%%%%%%%%%%%%%%%%%%%%%%%%%%%%%%%%%%%%%%%%%%%%%%%%%%%%%%%%%%%%%%%%
% SWITCHES                                                                  %
%%%%%%%%%%%%%%%%%%%%%%%%%%%%%%%%%%%%%%%%%%%%%%%%%%%%%%%%%%%%%%%%%%%%%%%%%%%%%
\newboolean{shading} 
\setboolean{shading}{false}
\makeatletter
 %% this is needed only when inserted into the file, not when
 %% used as a package file.
%%%%%%%%%%%%%%%%%%%%%%%%%%%%%%%%%%%%%%%%%%%%%%%%%%%%%%%%%%%%%%%%%%%%%%%%%%%%%
%                                                                           %
% DEFINITIONS OF SYMBOL-PRODUCING COMMANDS                                  %
%                                                                           %
%    TLA+      LaTeX                                                        %
%    symbol    command                                                      %
%    ------    -------                                                      %
%    =>        \implies                                                     %
%    <:        \ltcolon                                                     %
%    :>        \colongt                                                     %
%    ==        \defeq                                                       %
%    ..        \dotdot                                                      %
%    ::        \coloncolon                                                  %
%    =|        \eqdash                                                      %
%    ++        \pp                                                          %
%    --        \mm                                                          %
%    **        \stst                                                        %
%    //        \slsl                                                        %
%    ^         \ct                                                          %
%    \A        \A                                                           %
%    \E        \E                                                           %
%    \AA       \AA                                                          %
%    \EE       \EE                                                          %
%%%%%%%%%%%%%%%%%%%%%%%%%%%%%%%%%%%%%%%%%%%%%%%%%%%%%%%%%%%%%%%%%%%%%%%%%%%%%
\newlength{\symlength}
\newcommand{\implies}{\Rightarrow}
\newcommand{\ltcolon}{\mathrel{<\!\!\mbox{:}}}
\newcommand{\colongt}{\mathrel{\!\mbox{:}\!\!>}}
\newcommand{\defeq}{\;\mathrel{\smash   %% keep this symbol from being too tall
    {{\stackrel{\scriptscriptstyle\Delta}{=}}}}\;}
\newcommand{\dotdot}{\mathrel{\ldotp\ldotp}}
\newcommand{\coloncolon}{\mathrel{::\;}}
\newcommand{\eqdash}{\mathrel = \joinrel \hspace{-.28em}|}
\newcommand{\pp}{\mathbin{++}}
\newcommand{\mm}{\mathbin{--}}
\newcommand{\stst}{*\!*}
\newcommand{\slsl}{/\!/}
\newcommand{\ct}{\hat{\hspace{.4em}}}
\newcommand{\A}{\forall}
\newcommand{\E}{\exists}
\renewcommand{\AA}{\makebox{$\raisebox{.05em}{\makebox[0pt][l]{%
   $\forall\hspace{-.517em}\forall\hspace{-.517em}\forall$}}%
   \forall\hspace{-.517em}\forall \hspace{-.517em}\forall\,$}}
\newcommand{\EE}{\makebox{$\raisebox{.05em}{\makebox[0pt][l]{%
   $\exists\hspace{-.517em}\exists\hspace{-.517em}\exists$}}%
   \exists\hspace{-.517em}\exists\hspace{-.517em}\exists\,$}}
\newcommand{\whileop}{\.{\stackrel
  {\mbox{\raisebox{-.3em}[0pt][0pt]{$\scriptscriptstyle+\;\,$}}}%
  {-\hspace{-.16em}\triangleright}}}

% Commands are defined to produce the upper-case keywords.
% Note that some have space after them.
\newcommand{\ASSUME}{\textsc{assume }}
\newcommand{\ASSUMPTION}{\textsc{assumption }}
\newcommand{\AXIOM}{\textsc{axiom }}
\newcommand{\BOOLEAN}{\textsc{boolean }}
\newcommand{\CASE}{\textsc{case }}
\newcommand{\CONSTANT}{\textsc{constant }}
\newcommand{\CONSTANTS}{\textsc{constants }}
\newcommand{\ELSE}{\settowidth{\symlength}{\THEN}%
   \makebox[\symlength][l]{\textsc{ else}}}
\newcommand{\EXCEPT}{\textsc{ except }}
\newcommand{\EXTENDS}{\textsc{extends }}
\newcommand{\FALSE}{\textsc{false}}
\newcommand{\IF}{\textsc{if }}
\newcommand{\IN}{\settowidth{\symlength}{\LET}%
   \makebox[\symlength][l]{\textsc{in}}}
\newcommand{\INSTANCE}{\textsc{instance }}
\newcommand{\LET}{\textsc{let }}
\newcommand{\LOCAL}{\textsc{local }}
\newcommand{\MODULE}{\textsc{module }}
\newcommand{\OTHER}{\textsc{other }}
\newcommand{\STRING}{\textsc{string}}
\newcommand{\THEN}{\textsc{ then }}
\newcommand{\THEOREM}{\textsc{theorem }}
\newcommand{\LEMMA}{\textsc{lemma }}
\newcommand{\PROPOSITION}{\textsc{proposition }}
\newcommand{\COROLLARY}{\textsc{corollary }}
\newcommand{\TRUE}{\textsc{true}}
\newcommand{\VARIABLE}{\textsc{variable }}
\newcommand{\VARIABLES}{\textsc{variables }}
\newcommand{\WITH}{\textsc{ with }}
\newcommand{\WF}{\textrm{WF}}
\newcommand{\SF}{\textrm{SF}}
\newcommand{\CHOOSE}{\textsc{choose }}
\newcommand{\ENABLED}{\textsc{enabled }}
\newcommand{\UNCHANGED}{\textsc{unchanged }}
\newcommand{\SUBSET}{\textsc{subset }}
\newcommand{\UNION}{\textsc{union }}
\newcommand{\DOMAIN}{\textsc{domain }}
% Added for tla2tex
\newcommand{\BY}{\textsc{by }}
\newcommand{\OBVIOUS}{\textsc{obvious }}
\newcommand{\HAVE}{\textsc{have }}
\newcommand{\QED}{\textsc{qed }}
\newcommand{\TAKE}{\textsc{take }}
\newcommand{\DEF}{\textsc{ def }}
\newcommand{\HIDE}{\textsc{hide }}
\newcommand{\RECURSIVE}{\textsc{recursive }}
\newcommand{\USE}{\textsc{use }}
\newcommand{\DEFINE}{\textsc{define }}
\newcommand{\PROOF}{\textsc{proof }}
\newcommand{\WITNESS}{\textsc{witness }}
\newcommand{\PICK}{\textsc{pick }}
\newcommand{\DEFS}{\textsc{defs }}
\newcommand{\PROVE}{\settowidth{\symlength}{\ASSUME}%
   \makebox[\symlength][l]{\textsc{prove}}\@s{-4.1}}%
  %% The \@s{-4.1) is a kludge added on 24 Oct 2009 [happy birthday, Ellen]
  %% so the correct alignment occurs if the user types
  %%   ASSUME X
  %%   PROVE  Y
  %% because it cancels the extra 4.1 pts added because of the 
  %% extra space after the PROVE.  This seems to works OK.
  %% However, the 4.1 equals Parameters.LaTeXLeftSpace(1) and
  %% should be changed if that method ever changes.
\newcommand{\SUFFICES}{\textsc{suffices }}
\newcommand{\NEW}{\textsc{new }}
\newcommand{\LAMBDA}{\textsc{lambda }}
\newcommand{\STATE}{\textsc{state }}
\newcommand{\ACTION}{\textsc{action }}
\newcommand{\TEMPORAL}{\textsc{temporal }}
\newcommand{\ONLY}{\textsc{only }}              %% added by LL on 2 Oct 2009
\newcommand{\OMITTED}{\textsc{omitted }}        %% added by LL on 31 Oct 2009
\newcommand{\@pfstepnum}[2]{\ensuremath{\langle#1\rangle}\textrm{#2}}
\newcommand{\bang}{\@s{1}\mbox{\small !}\@s{1}}
%% We should format || differently in PlusCal code than in TLA+ formulas.
\newcommand{\p@barbar}{\ifpcalsymbols
   \,\,\rule[-.25em]{.075em}{1em}\hspace*{.2em}\rule[-.25em]{.075em}{1em}\,\,%
   \else \,||\,\fi}
%% PlusCal keywords
\newcommand{\p@fair}{\textbf{fair }}
\newcommand{\p@semicolon}{\textbf{\,; }}
\newcommand{\p@algorithm}{\textbf{algorithm }}
\newcommand{\p@mmfair}{\textbf{-{}-fair }}
\newcommand{\p@mmalgorithm}{\textbf{-{}-algorithm }}
\newcommand{\p@assert}{\textbf{assert }}
\newcommand{\p@await}{\textbf{await }}
\newcommand{\p@begin}{\textbf{begin }}
\newcommand{\p@end}{\textbf{end }}
\newcommand{\p@call}{\textbf{call }}
\newcommand{\p@define}{\textbf{define }}
\newcommand{\p@do}{\textbf{ do }}
\newcommand{\p@either}{\textbf{either }}
\newcommand{\p@or}{\textbf{or }}
\newcommand{\p@goto}{\textbf{goto }}
\newcommand{\p@if}{\textbf{if }}
\newcommand{\p@then}{\,\,\textbf{then }}
\newcommand{\p@else}{\ifcsyntax \textbf{else } \else \,\,\textbf{else }\fi}
\newcommand{\p@elsif}{\,\,\textbf{elsif }}
\newcommand{\p@macro}{\textbf{macro }}
\newcommand{\p@print}{\textbf{print }}
\newcommand{\p@procedure}{\textbf{procedure }}
\newcommand{\p@process}{\textbf{process }}
\newcommand{\p@return}{\textbf{return}}
\newcommand{\p@skip}{\textbf{skip}}
\newcommand{\p@variable}{\textbf{variable }}
\newcommand{\p@variables}{\textbf{variables }}
\newcommand{\p@while}{\textbf{while }}
\newcommand{\p@when}{\textbf{when }}
\newcommand{\p@with}{\textbf{with }}
\newcommand{\p@lparen}{\textbf{(\,\,}}
\newcommand{\p@rparen}{\textbf{\,\,) }}   
\newcommand{\p@lbrace}{\textbf{\{\,\,}}   
\newcommand{\p@rbrace}{\textbf{\,\,\} }}

%%%%%%%%%%%%%%%%%%%%%%%%%%%%%%%%%%%%%%%%%%%%%%%%%%%%%%%%%
% REDEFINE STANDARD COMMANDS TO MAKE THEM FORMAT BETTER %
%                                                       %
% We redefine \in and \notin                            %
%%%%%%%%%%%%%%%%%%%%%%%%%%%%%%%%%%%%%%%%%%%%%%%%%%%%%%%%%
\renewcommand{\_}{\rule{.4em}{.06em}\hspace{.05em}}
\newlength{\equalswidth}
\let\oldin=\in
\let\oldnotin=\notin
\renewcommand{\in}{%
   {\settowidth{\equalswidth}{$\.{=}$}\makebox[\equalswidth][c]{$\oldin$}}}
\renewcommand{\notin}{%
   {\settowidth{\equalswidth}{$\.{=}$}\makebox[\equalswidth]{$\oldnotin$}}}


%%%%%%%%%%%%%%%%%%%%%%%%%%%%%%%%%%%%%%%%%%%%%%%%%%%%
%                                                  %
% HORIZONTAL BARS:                                 %
%                                                  %
%   \moduleLeftDash    |~~~~~~~~~~                 %
%   \moduleRightDash    ~~~~~~~~~~|                %
%   \midbar            |----------|                %
%   \bottombar         |__________|                %
%%%%%%%%%%%%%%%%%%%%%%%%%%%%%%%%%%%%%%%%%%%%%%%%%%%%
\newlength{\charwidth}\settowidth{\charwidth}{{\small\tt M}}
\newlength{\boxrulewd}\setlength{\boxrulewd}{.4pt}
\newlength{\boxlineht}\setlength{\boxlineht}{.5\baselineskip}
\newcommand{\boxsep}{\charwidth}
\newlength{\boxruleht}\setlength{\boxruleht}{.5ex}
\newlength{\boxruledp}\setlength{\boxruledp}{-\boxruleht}
\addtolength{\boxruledp}{\boxrulewd}
\newcommand{\boxrule}{\leaders\hrule height \boxruleht depth \boxruledp
                      \hfill\mbox{}}
\newcommand{\@computerule}{%
  \setlength{\boxruleht}{.5ex}%
  \setlength{\boxruledp}{-\boxruleht}%
  \addtolength{\boxruledp}{\boxrulewd}}

\newcommand{\bottombar}{\hspace{-\boxsep}%
  \raisebox{-\boxrulewd}[0pt][0pt]{\rule[.5ex]{\boxrulewd}{\boxlineht}}%
  \boxrule
  \raisebox{-\boxrulewd}[0pt][0pt]{%
      \rule[.5ex]{\boxrulewd}{\boxlineht}}\hspace{-\boxsep}\vspace{0pt}}

\newcommand{\moduleLeftDash}%
   {\hspace*{-\boxsep}%
     \raisebox{-\boxlineht}[0pt][0pt]{\rule[.5ex]{\boxrulewd
               }{\boxlineht}}%
    \boxrule\hspace*{.4em }}

\newcommand{\moduleRightDash}%
    {\hspace*{.4em}\boxrule
    \raisebox{-\boxlineht}[0pt][0pt]{\rule[.5ex]{\boxrulewd
               }{\boxlineht}}\hspace{-\boxsep}}%\vspace{.2em}

\newcommand{\midbar}{\hspace{-\boxsep}\raisebox{-.5\boxlineht}[0pt][0pt]{%
   \rule[.5ex]{\boxrulewd}{\boxlineht}}\boxrule\raisebox{-.5\boxlineht%
   }[0pt][0pt]{\rule[.5ex]{\boxrulewd}{\boxlineht}}\hspace{-\boxsep}}

%%%%%%%%%%%%%%%%%%%%%%%%%%%%%%%%%%%%%%%%%%%%%%%%%%%%%%%%%%%%%%%%%%%%%%%%%%%%%
% FORMATING COMMANDS                                                        %
%%%%%%%%%%%%%%%%%%%%%%%%%%%%%%%%%%%%%%%%%%%%%%%%%%%%%%%%%%%%%%%%%%%%%%%%%%%%%

%%%%%%%%%%%%%%%%%%%%%%%%%%%%%%%%%%%%%%%%%%%%%%%%%%%%%%%%%%%%%%%%%%%%%%%%%%%%%
% PLUSCAL SHADING                                                           %
%%%%%%%%%%%%%%%%%%%%%%%%%%%%%%%%%%%%%%%%%%%%%%%%%%%%%%%%%%%%%%%%%%%%%%%%%%%%%

% The TeX pcalshading switch is set on to cause PlusCal shading to be
% performed.  This changes the behavior of the following commands and
% environments to cause full-width shading to be performed on all lines.
% 
%   \tstrut \@x cpar mcom \@pvspace
% 
% The TeX pcalsymbols switch is turned on when typesetting a PlusCal algorithm,
% whether or not shading is being performed.  It causes symbols (other than
% parentheses and braces and PlusCal-only keywords) that should be typeset
% differently depending on whether they are in an algorithm to be typeset
% appropriately.  Currently, the only such symbol is "||".
%
% The TeX csyntax switch is turned on when typesetting a PlusCal algorithm in
% c-syntax.  This allows symbols to be format differently in the two syntaxes.
% The "else" keyword is the only one that is.

\newif\ifpcalshading \pcalshadingfalse
\newif\ifpcalsymbols \pcalsymbolsfalse
\newif\ifcsyntax     \csyntaxtrue

% The \@pvspace command makes a vertical space.  It uses \vspace
% except with \ifpcalshading, in which case it sets \pvcalvspace
% and the space is added by a following \@x command.
%
\newlength{\pcalvspace}\setlength{\pcalvspace}{0pt}%
\newcommand{\@pvspace}[1]{%
  \ifpcalshading
     \par\global\setlength{\pcalvspace}{#1}%
  \else
     \par\vspace{#1}%
  \fi
}

% The lcom environment was changed to set \lcomindent equal to
% the indentation it produces.  This length is used by the
% cpar environment to make shading extend for the full width
% of the line.  This assumes that lcom environments are not
% nested.  I hope TLATeX does not nest them.
%
\newlength{\lcomindent}%
\setlength{\lcomindent}{0pt}%

%\tstrut: A strut to produce inter-paragraph space in a comment.
%\rstrut: A strut to extend the bottom of a one-line comment so
%         there's no break in the shading between comments on 
%         successive lines.
\newcommand\tstrut%
  {\raisebox{\vshadelen}{\raisebox{-.25em}{\rule{0pt}{1.15em}}}%
   \global\setlength{\vshadelen}{0pt}}
\newcommand\rstrut{\raisebox{-.25em}{\rule{0pt}{1.15em}}%
 \global\setlength{\vshadelen}{0pt}}


% \.{op} formats operator op in math mode with empty boxes on either side.
% Used because TeX otherwise vary the amount of space it leaves around op.
\renewcommand{\.}[1]{\ensuremath{\mbox{}#1\mbox{}}}

% \@s{n} produces an n-point space
\newcommand{\@s}[1]{\hspace{#1pt}}           

% \@x{txt} starts a specification line in the beginning with txt
% in the final LaTeX source.
\newlength{\@xlen}
\newcommand\xtstrut%
  {\setlength{\@xlen}{1.05em}%
   \addtolength{\@xlen}{\pcalvspace}%
    \raisebox{\vshadelen}{\raisebox{-.25em}{\rule{0pt}{\@xlen}}}%
   \global\setlength{\vshadelen}{0pt}%
   \global\setlength{\pcalvspace}{0pt}}

\newcommand{\@x}[1]{\par
  \ifpcalshading
  \makebox[0pt][l]{\shadebox{\xtstrut\hspace*{\textwidth}}}%
  \fi
  \mbox{$\mbox{}#1\mbox{}$}}  

% \@xx{txt} continues a specification line with the text txt.
\newcommand{\@xx}[1]{\mbox{$\mbox{}#1\mbox{}$}}  

% \@y{cmt} produces a one-line comment.
\newcommand{\@y}[1]{\mbox{\footnotesize\hspace{.65em}%
  \ifthenelse{\boolean{shading}}{%
      \shadebox{#1\hspace{-\the\lastskip}\rstrut}}%
               {#1\hspace{-\the\lastskip}\rstrut}}}

% \@z{cmt} produces a zero-width one-line comment.
\newcommand{\@z}[1]{\makebox[0pt][l]{\footnotesize
  \ifthenelse{\boolean{shading}}{%
      \shadebox{#1\hspace{-\the\lastskip}\rstrut}}%
               {#1\hspace{-\the\lastskip}\rstrut}}}


% \@w{str} produces the TLA+ string "str".
\newcommand{\@w}[1]{\textsf{``{#1}''}}             


%%%%%%%%%%%%%%%%%%%%%%%%%%%%%%%%%%%%%%%%%%%%%%%%%%%%%%%%%%%%%%%%%%%%%%%%%%%%%
% SHADING                                                                   %
%%%%%%%%%%%%%%%%%%%%%%%%%%%%%%%%%%%%%%%%%%%%%%%%%%%%%%%%%%%%%%%%%%%%%%%%%%%%%
\def\graymargin{1}
  % The number of points of margin in the shaded box.

% \definecolor{boxshade}{gray}{.85}
% Defines the darkness of the shading: 1 = white, 0 = black
% Added by TLATeX only if needed.

% \shadebox{txt} puts txt in a shaded box.
\newlength{\templena}
\newlength{\templenb}
\newsavebox{\tempboxa}
\newcommand{\shadebox}[1]{{\setlength{\fboxsep}{\graymargin pt}%
     \savebox{\tempboxa}{#1}%
     \settoheight{\templena}{\usebox{\tempboxa}}%
     \settodepth{\templenb}{\usebox{\tempboxa}}%
     \hspace*{-\fboxsep}\raisebox{0pt}[\templena][\templenb]%
        {\colorbox{boxshade}{\usebox{\tempboxa}}}\hspace*{-\fboxsep}}}

% \vshade{n} makes an n-point inter-paragraph space, with
%  shading if the `shading' flag is true.
\newlength{\vshadelen}
\setlength{\vshadelen}{0pt}
\newcommand{\vshade}[1]{\ifthenelse{\boolean{shading}}%
   {\global\setlength{\vshadelen}{#1pt}}%
   {\vspace{#1pt}}}

\newlength{\boxwidth}
\newlength{\multicommentdepth}

%%%%%%%%%%%%%%%%%%%%%%%%%%%%%%%%%%%%%%%%%%%%%%%%%%%%%%%%%%%%%%%%%%%%%%%%%%%%%
% THE cpar ENVIRONMENT                                                      %
% ^^^^^^^^^^^^^^^^^^^^                                                      %
% The LaTeX input                                                           %
%                                                                           %
%   \begin{cpar}{pop}{nest}{isLabel}{d}{e}{arg6}                            %
%     XXXXXXXXXXXXXXX                                                       %
%     XXXXXXXXXXXXXXX                                                       %
%     XXXXXXXXXXXXXXX                                                       %
%   \end{cpar}                                                              %
%                                                                           %
% produces one of two possible results.  If isLabel is the letter "T",      %
% it produces the following, where [label] is the result of typesetting     %
% arg6 in an LR box, and d is is a number representing a distance in        %
% points.                                                                   %
%                                                                           %
%   prevailing |<-- d -->[label]<- e ->XXXXXXXXXXXXXXX                      %
%         left |                       XXXXXXXXXXXXXXX                      %
%       margin |                       XXXXXXXXXXXXXXX                      %
%                                                                           %
% If isLabel is the letter "F", then it produces                            %
%                                                                           %
%   prevailing |<-- d -->XXXXXXXXXXXXXXXXXXXXXXX                            %
%         left |         <- e ->XXXXXXXXXXXXXXXX                            %
%       margin |                XXXXXXXXXXXXXXXX                            %
%                                                                           %
% where d and e are numbers representing distances in points.               %
%                                                                           %
% The prevailing left margin is the one in effect before the most recent    %
% pop (argument 1) cpar environments with "T" as the nest argument, where   %
% pop is a number \geq 0.                                                   %
%                                                                           %
% If the nest argument is the letter "T", then the prevailing left          %
% margin is moved to the left of the second (and following) lines of        %
% X's.  Otherwise, the prevailing left margin is left unchanged.            %
%                                                                           %
% An \unnest{n} command moves the prevailing left margin to where it was    %
% before the most recent n cpar environments with "T" as the nesting        %
% argument.                                                                 %
%                                                                           %
% The environment leaves no vertical space above or below it, or between    %
% its paragraphs.  (TLATeX inserts the proper amount of vertical space.)    %
%%%%%%%%%%%%%%%%%%%%%%%%%%%%%%%%%%%%%%%%%%%%%%%%%%%%%%%%%%%%%%%%%%%%%%%%%%%%%

\newcounter{pardepth}
\setcounter{pardepth}{0}

% \setgmargin{txt} defines \gmarginN to be txt, where N is \roman{pardepth}.
% \thegmargin equals \gmarginN, where N is \roman{pardepth}.
\newcommand{\setgmargin}[1]{%
  \expandafter\xdef\csname gmargin\roman{pardepth}\endcsname{#1}}
\newcommand{\thegmargin}{\csname gmargin\roman{pardepth}\endcsname}
\newcommand{\gmargin}{0pt}

\newsavebox{\tempsbox}

\newlength{\@cparht}
\newlength{\@cpardp}
\newenvironment{cpar}[6]{%
  \addtocounter{pardepth}{-#1}%
  \ifthenelse{\boolean{shading}}{\par\begin{lrbox}{\tempsbox}%
                                 \begin{minipage}[t]{\linewidth}}{}%
  \begin{list}{}{%
     \edef\temp{\thegmargin}
     \ifthenelse{\equal{#3}{T}}%
       {\settowidth{\leftmargin}{\hspace{\temp}\footnotesize #6\hspace{#5pt}}%
        \addtolength{\leftmargin}{#4pt}}%
       {\setlength{\leftmargin}{#4pt}%
        \addtolength{\leftmargin}{#5pt}%
        \addtolength{\leftmargin}{\temp}%
        \setlength{\itemindent}{-#5pt}}%
      \ifthenelse{\equal{#2}{T}}{\addtocounter{pardepth}{1}%
                                 \setgmargin{\the\leftmargin}}{}%
      \setlength{\labelwidth}{0pt}%
      \setlength{\labelsep}{0pt}%
      \setlength{\itemindent}{-\leftmargin}%
      \setlength{\topsep}{0pt}%
      \setlength{\parsep}{0pt}%
      \setlength{\partopsep}{0pt}%
      \setlength{\parskip}{0pt}%
      \setlength{\itemsep}{0pt}
      \setlength{\itemindent}{#4pt}%
      \addtolength{\itemindent}{-\leftmargin}}%
   \ifthenelse{\equal{#3}{T}}%
      {\item[\tstrut\footnotesize \hspace{\temp}{#6}\hspace{#5pt}]
        }%
      {\item[\tstrut\hspace{\temp}]%
         }%
   \footnotesize}
 {\hspace{-\the\lastskip}\tstrut
 \end{list}%
  \ifthenelse{\boolean{shading}}%
          {\end{minipage}%
           \end{lrbox}%
           \ifpcalshading
             \setlength{\@cparht}{\ht\tempsbox}%
             \setlength{\@cpardp}{\dp\tempsbox}%
             \addtolength{\@cparht}{.15em}%
             \addtolength{\@cpardp}{.2em}%
             \addtolength{\@cparht}{\@cpardp}%
            % I don't know what's going on here.  I want to add a
            % \pcalvspace high shaded line, but I don't know how to
            % do it.  A little trial and error shows that the following
            % does a reasonable job approximating that, eliminating
            % the line if \pcalvspace is small.
            \addtolength{\@cparht}{\pcalvspace}%
             \ifdim \pcalvspace > .8em
               \addtolength{\pcalvspace}{-.2em}%
               \hspace*{-\lcomindent}%
               \shadebox{\rule{0pt}{\pcalvspace}\hspace*{\textwidth}}\par
               \global\setlength{\pcalvspace}{0pt}%
               \fi
             \hspace*{-\lcomindent}%
             \makebox[0pt][l]{\raisebox{-\@cpardp}[0pt][0pt]{%
                 \shadebox{\rule{0pt}{\@cparht}\hspace*{\textwidth}}}}%
             \hspace*{\lcomindent}\usebox{\tempsbox}%
             \par
           \else
             \shadebox{\usebox{\tempsbox}}\par
           \fi}%
           {}%
  }

%%%%%%%%%%%%%%%%%%%%%%%%%%%%%%%%%%%%%%%%%%%%%%%%%%%%%%%%%%%%%%%%%%%%%%%%%%%%%%
% THE ppar ENVIRONMENT                                                       %
% ^^^^^^^^^^^^^^^^^^^^                                                       %
% The environment                                                            %
%                                                                            %
%   \begin{ppar} ... \end{ppar}                                              %
%                                                                            %
% is equivalent to                                                           %
%                                                                            %
%   \begin{cpar}{0}{F}{F}{0}{0}{} ... \end{cpar}                             %
%                                                                            %
% The environment is put around each line of the output for a PlusCal        %
% algorithm.                                                                 %
%%%%%%%%%%%%%%%%%%%%%%%%%%%%%%%%%%%%%%%%%%%%%%%%%%%%%%%%%%%%%%%%%%%%%%%%%%%%%%
%\newenvironment{ppar}{%
%  \ifthenelse{\boolean{shading}}{\par\begin{lrbox}{\tempsbox}%
%                                 \begin{minipage}[t]{\linewidth}}{}%
%  \begin{list}{}{%
%     \edef\temp{\thegmargin}
%        \setlength{\leftmargin}{0pt}%
%        \addtolength{\leftmargin}{\temp}%
%        \setlength{\itemindent}{0pt}%
%      \setlength{\labelwidth}{0pt}%
%      \setlength{\labelsep}{0pt}%
%      \setlength{\itemindent}{-\leftmargin}%
%      \setlength{\topsep}{0pt}%
%      \setlength{\parsep}{0pt}%
%      \setlength{\partopsep}{0pt}%
%      \setlength{\parskip}{0pt}%
%      \setlength{\itemsep}{0pt}
%      \setlength{\itemindent}{0pt}%
%      \addtolength{\itemindent}{-\leftmargin}}%
%      \item[\tstrut\hspace{\temp}]}%
% {\hspace{-\the\lastskip}\tstrut
% \end{list}%
%  \ifthenelse{\boolean{shading}}{\end{minipage}  
%                                 \end{lrbox}%
%                                 \shadebox{\usebox{\tempsbox}}\par}{}%
%  }

 %%% TESTING
 \newcommand{\xtest}[1]{\par
 \makebox[0pt][l]{\shadebox{\xtstrut\hspace*{\textwidth}}}%
 \mbox{$\mbox{}#1\mbox{}$}} 

% \newcommand{\xxtest}[1]{\par
% \makebox[0pt][l]{\shadebox{\xtstrut{#1}\hspace*{\textwidth}}}%
% \mbox{$\mbox{}#1\mbox{}$}} 

%\newlength{\pcalvspace}
%\setlength{\pcalvspace}{0pt}
% \newlength{\xxtestlen}
% \setlength{\xxtestlen}{0pt}
% \newcommand\xtstrut%
%   {\setlength{\xxtestlen}{1.15em}%
%    \addtolength{\xxtestlen}{\pcalvspace}%
%     \raisebox{\vshadelen}{\raisebox{-.25em}{\rule{0pt}{\xxtestlen}}}%
%    \global\setlength{\vshadelen}{0pt}%
%    \global\setlength{\pcalvspace}{0pt}}
   
   %%%% TESTING
   
   %% The xcpar environment
   %%  Note: overloaded use of \pcalvspace for testing.
   %%
%   \newlength{\xcparht}%
%   \newlength{\xcpardp}%
   
%   \newenvironment{xcpar}[6]{%
%  \addtocounter{pardepth}{-#1}%
%  \ifthenelse{\boolean{shading}}{\par\begin{lrbox}{\tempsbox}%
%                                 \begin{minipage}[t]{\linewidth}}{}%
%  \begin{list}{}{%
%     \edef\temp{\thegmargin}%
%     \ifthenelse{\equal{#3}{T}}%
%       {\settowidth{\leftmargin}{\hspace{\temp}\footnotesize #6\hspace{#5pt}}%
%        \addtolength{\leftmargin}{#4pt}}%
%       {\setlength{\leftmargin}{#4pt}%
%        \addtolength{\leftmargin}{#5pt}%
%        \addtolength{\leftmargin}{\temp}%
%        \setlength{\itemindent}{-#5pt}}%
%      \ifthenelse{\equal{#2}{T}}{\addtocounter{pardepth}{1}%
%                                 \setgmargin{\the\leftmargin}}{}%
%      \setlength{\labelwidth}{0pt}%
%      \setlength{\labelsep}{0pt}%
%      \setlength{\itemindent}{-\leftmargin}%
%      \setlength{\topsep}{0pt}%
%      \setlength{\parsep}{0pt}%
%      \setlength{\partopsep}{0pt}%
%      \setlength{\parskip}{0pt}%
%      \setlength{\itemsep}{0pt}%
%      \setlength{\itemindent}{#4pt}%
%      \addtolength{\itemindent}{-\leftmargin}}%
%   \ifthenelse{\equal{#3}{T}}%
%      {\item[\xtstrut\footnotesize \hspace{\temp}{#6}\hspace{#5pt}]%
%        }%
%      {\item[\xtstrut\hspace{\temp}]%
%         }%
%   \footnotesize}
% {\hspace{-\the\lastskip}\tstrut
% \end{list}%
%  \ifthenelse{\boolean{shading}}{\end{minipage}  
%                                 \end{lrbox}%
%                                 \setlength{\xcparht}{\ht\tempsbox}%
%                                 \setlength{\xcpardp}{\dp\tempsbox}%
%                                 \addtolength{\xcparht}{.15em}%
%                                 \addtolength{\xcpardp}{.2em}%
%                                 \addtolength{\xcparht}{\xcpardp}%
%                                 \hspace*{-\lcomindent}%
%                                 \makebox[0pt][l]{\raisebox{-\xcpardp}[0pt][0pt]{%
%                                      \shadebox{\rule{0pt}{\xcparht}\hspace*{\textwidth}}}}%
%                                 \hspace*{\lcomindent}\usebox{\tempsbox}%
%                                 \par}{}%
%  }
%  
% \newlength{\xmcomlen}
%\newenvironment{xmcom}[1]{%
%  \setcounter{pardepth}{0}%
%  \hspace{.65em}%
%  \begin{lrbox}{\alignbox}\sloppypar%
%      \setboolean{shading}{false}%
%      \setlength{\boxwidth}{#1pt}%
%      \addtolength{\boxwidth}{-.65em}%
%      \begin{minipage}[t]{\boxwidth}\footnotesize
%      \parskip=0pt\relax}%
%       {\end{minipage}\end{lrbox}%
%       \setlength{\xmcomlen}{\textwidth}%
%       \addtolength{\xmcomlen}{-\wd\alignbox}%
%       \settodepth{\alignwidth}{\usebox{\alignbox}}%
%       \global\setlength{\multicommentdepth}{\alignwidth}%
%       \setlength{\boxwidth}{\alignwidth}%
%       \global\addtolength{\alignwidth}{-\maxdepth}%
%       \addtolength{\boxwidth}{.1em}%
%       \raisebox{0pt}[0pt][0pt]{%
%        \ifthenelse{\boolean{shading}}%
%          {\hspace*{-\xmcomlen}\shadebox{\rule[-\boxwidth]{0pt}{0pt}%
%                                 \hspace*{\xmcomlen}\usebox{\alignbox}}}%
%          {\usebox{\alignbox}}}%
%       \vspace*{\alignwidth}\pagebreak[0]\vspace{-\alignwidth}\par}
% % a multi-line comment, whose first argument is its width in points.
%  
   
%%%%%%%%%%%%%%%%%%%%%%%%%%%%%%%%%%%%%%%%%%%%%%%%%%%%%%%%%%%%%%%%%%%%%%%%%%%%%%
% THE lcom ENVIRONMENT                                                       %
% ^^^^^^^^^^^^^^^^^^^^                                                       %
% A multi-line comment with no text to its left is typeset in an lcom        % 
% environment, whose argument is a number representing the indentation       % 
% of the left margin, in points.  All the text of the comment should be      % 
% inside cpar environments.                                                  % 
%%%%%%%%%%%%%%%%%%%%%%%%%%%%%%%%%%%%%%%%%%%%%%%%%%%%%%%%%%%%%%%%%%%%%%%%%%%%%%
\newenvironment{lcom}[1]{%
  \setlength{\lcomindent}{#1pt} % Added for PlusCal handling.
  \par\vspace{.2em}%
  \sloppypar
  \setcounter{pardepth}{0}%
  \footnotesize
  \begin{list}{}{%
    \setlength{\leftmargin}{#1pt}
    \setlength{\labelwidth}{0pt}%
    \setlength{\labelsep}{0pt}%
    \setlength{\itemindent}{0pt}%
    \setlength{\topsep}{0pt}%
    \setlength{\parsep}{0pt}%
    \setlength{\partopsep}{0pt}%
    \setlength{\parskip}{0pt}}
    \item[]}%
  {\end{list}\vspace{.3em}\setlength{\lcomindent}{0pt}%
 }


%%%%%%%%%%%%%%%%%%%%%%%%%%%%%%%%%%%%%%%%%%%%%%%%%%%%%%%%%%%%%%%%%%%%%%%%%%%%%
% THE mcom ENVIRONMENT AND \mutivspace COMMAND                              %
% ^^^^^^^^^^^^^^^^^^^^^^^^^^^^^^^^^^^^^^^^^^^^                              %
%                                                                           %
% A part of the spec containing a right-comment of the form                 %
%                                                                           %
%      xxxx (*************)                                                 %
%      yyyy (* ccccccccc *)                                                 %
%      ...  (* ccccccccc *)                                                 %
%           (* ccccccccc *)                                                 %
%           (* ccccccccc *)                                                 %
%           (*************)                                                 %
%                                                                           %
% is typeset by                                                             %
%                                                                           %
%     XXXX \begin{mcom}{d}                                                  %
%            CCCC ... CCC                                                   %
%          \end{mcom}                                                       %
%     YYYY ...                                                              %
%     \multivspace{n}                                                       %
%                                                                           %
% where the number d is the width in points of the comment, n is the        %
% number of xxxx, yyyy, ...  lines to the left of the comment.              %
% All the text of the comment should be typeset in cpar environments.       %
%                                                                           %
% This puts the comment into a single box (so no page breaks can occur      %
% within it).  The entire box is shaded iff the shading flag is true.       %
%%%%%%%%%%%%%%%%%%%%%%%%%%%%%%%%%%%%%%%%%%%%%%%%%%%%%%%%%%%%%%%%%%%%%%%%%%%%%
\newlength{\xmcomlen}%
\newenvironment{mcom}[1]{%
  \setcounter{pardepth}{0}%
  \hspace{.65em}%
  \begin{lrbox}{\alignbox}\sloppypar%
      \setboolean{shading}{false}%
      \setlength{\boxwidth}{#1pt}%
      \addtolength{\boxwidth}{-.65em}%
      \begin{minipage}[t]{\boxwidth}\footnotesize
      \parskip=0pt\relax}%
       {\end{minipage}\end{lrbox}%
       \setlength{\xmcomlen}{\textwidth}%       % For PlusCal shading
       \addtolength{\xmcomlen}{-\wd\alignbox}%  % For PlusCal shading
       \settodepth{\alignwidth}{\usebox{\alignbox}}%
       \global\setlength{\multicommentdepth}{\alignwidth}%
       \setlength{\boxwidth}{\alignwidth}%      % For PlusCal shading
       \global\addtolength{\alignwidth}{-\maxdepth}%
       \addtolength{\boxwidth}{.1em}%           % For PlusCal shading
      \raisebox{0pt}[0pt][0pt]{%
        \ifthenelse{\boolean{shading}}%
          {\ifpcalshading
             \hspace*{-\xmcomlen}%
             \shadebox{\rule[-\boxwidth]{0pt}{0pt}\hspace*{\xmcomlen}%
                          \usebox{\alignbox}}%
           \else
             \shadebox{\usebox{\alignbox}}
           \fi
          }%
          {\usebox{\alignbox}}}%
       \vspace*{\alignwidth}\pagebreak[0]\vspace{-\alignwidth}\par}
 % a multi-line comment, whose first argument is its width in points.


% \multispace{n} produces the vertical space indicated by "|"s in 
% this situation
%   
%     xxxx (*************)
%     xxxx (* ccccccccc *)
%      |   (* ccccccccc *)
%      |   (* ccccccccc *)
%      |   (* ccccccccc *)
%      |   (*************)
%
% where n is the number of "xxxx" lines.
\newcommand{\multivspace}[1]{\addtolength{\multicommentdepth}{-#1\baselineskip}%
 \addtolength{\multicommentdepth}{1.2em}%
 \ifthenelse{\lengthtest{\multicommentdepth > 0pt}}%
    {\par\vspace{\multicommentdepth}\par}{}}

%\newenvironment{hpar}[2]{%
%  \begin{list}{}{\setlength{\leftmargin}{#1pt}%
%                 \addtolength{\leftmargin}{#2pt}%
%                 \setlength{\itemindent}{-#2pt}%
%                 \setlength{\topsep}{0pt}%
%                 \setlength{\parsep}{0pt}%
%                 \setlength{\partopsep}{0pt}%
%                 \setlength{\parskip}{0pt}%
%                 \addtolength{\labelsep}{0pt}}%
%  \item[]\footnotesize}{\end{list}}
%    %%%%%%%%%%%%%%%%%%%%%%%%%%%%%%%%%%%%%%%%%%%%%%%%%%%%%%%%%%%%%%%%%%%%%%%%
%    % Typesets a sequence of paragraphs like this:                         %
%    %                                                                      %
%    %      left |<-- d1 --> XXXXXXXXXXXXXXXXXXXXXXXX                       %
%    %    margin |           <- d2 -> XXXXXXXXXXXXXXX                       %
%    %           |                    XXXXXXXXXXXXXXX                       %
%    %           |                                                          %
%    %           |                    XXXXXXXXXXXXXXX                       %
%    %           |                    XXXXXXXXXXXXXXX                       %
%    %                                                                      %
%    % where d1 = #1pt and d2 = #2pt, but with no vspace between            %
%    % paragraphs.                                                          %
%    %%%%%%%%%%%%%%%%%%%%%%%%%%%%%%%%%%%%%%%%%%%%%%%%%%%%%%%%%%%%%%%%%%%%%%%%

%%%%%%%%%%%%%%%%%%%%%%%%%%%%%%%%%%%%%%%%%%%%%%%%%%%%%%%%%%%%%%%%%%%%%%
% Commands for repeated characters that produce dashes.              %
%%%%%%%%%%%%%%%%%%%%%%%%%%%%%%%%%%%%%%%%%%%%%%%%%%%%%%%%%%%%%%%%%%%%%%
% \raisedDash{wd}{ht}{thk} makes a horizontal line wd characters wide, 
% raised a distance ht ex's above the baseline, with a thickness of 
% thk em's.
\newcommand{\raisedDash}[3]{\raisebox{#2ex}{\setlength{\alignwidth}{.5em}%
  \rule{#1\alignwidth}{#3em}}}

% The following commands take a single argument n and produce the
% output for n repeated characters, as follows
%   \cdash:    -
%   \tdash:    ~
%   \ceqdash:  =
%   \usdash:   _
\newcommand{\cdash}[1]{\raisedDash{#1}{.5}{.04}}
\newcommand{\usdash}[1]{\raisedDash{#1}{0}{.04}}
\newcommand{\ceqdash}[1]{\raisedDash{#1}{.5}{.08}}
\newcommand{\tdash}[1]{\raisedDash{#1}{1}{.08}}

\newlength{\spacewidth}
\setlength{\spacewidth}{.2em}
\newcommand{\e}[1]{\hspace{#1\spacewidth}}
%% \e{i} produces space corresponding to i input spaces.


%% Alignment-file Commands

\newlength{\alignboxwidth}
\newlength{\alignwidth}
\newsavebox{\alignbox}

% \al{i}{j}{txt} is used in the alignment file to put "%{i}{j}{wd}"
% in the log file, where wd is the width of the line up to that point,
% and txt is the following text.
\newcommand{\al}[3]{%
  \typeout{\%{#1}{#2}{\the\alignwidth}}%
  \cl{#3}}

%% \cl{txt} continues a specification line in the alignment file
%% with text txt.
\newcommand{\cl}[1]{%
  \savebox{\alignbox}{\mbox{$\mbox{}#1\mbox{}$}}%
  \settowidth{\alignboxwidth}{\usebox{\alignbox}}%
  \addtolength{\alignwidth}{\alignboxwidth}%
  \usebox{\alignbox}}

% \fl{txt} in the alignment file begins a specification line that
% starts with the text txt.
\newcommand{\fl}[1]{%
  \par
  \savebox{\alignbox}{\mbox{$\mbox{}#1\mbox{}$}}%
  \settowidth{\alignwidth}{\usebox{\alignbox}}%
  \usebox{\alignbox}}



  
%%%%%%%%%%%%%%%%%%%%%%%%%%%%%%%%%%%%%%%%%%%%%%%%%%%%%%%%%%%%%%%%%%%%%%%%%%%%%
% Ordinarily, TeX typesets letters in math mode in a special math italic    %
% font.  This makes it typeset "it" to look like the product of the         %
% variables i and t, rather than like the word "it".  The following         %
% commands tell TeX to use an ordinary italic font instead.                 %
%%%%%%%%%%%%%%%%%%%%%%%%%%%%%%%%%%%%%%%%%%%%%%%%%%%%%%%%%%%%%%%%%%%%%%%%%%%%%
\ifx\documentclass\undefined
\else
  \DeclareSymbolFont{tlaitalics}{\encodingdefault}{cmr}{m}{it}
  \let\itfam\symtlaitalics
\fi

\makeatletter
\newcommand{\tlx@c}{\c@tlx@ctr\advance\c@tlx@ctr\@ne}
\newcounter{tlx@ctr}
\c@tlx@ctr=\itfam \multiply\c@tlx@ctr"100\relax \advance\c@tlx@ctr "7061\relax
\mathcode`a=\tlx@c \mathcode`b=\tlx@c \mathcode`c=\tlx@c \mathcode`d=\tlx@c
\mathcode`e=\tlx@c \mathcode`f=\tlx@c \mathcode`g=\tlx@c \mathcode`h=\tlx@c
\mathcode`i=\tlx@c \mathcode`j=\tlx@c \mathcode`k=\tlx@c \mathcode`l=\tlx@c
\mathcode`m=\tlx@c \mathcode`n=\tlx@c \mathcode`o=\tlx@c \mathcode`p=\tlx@c
\mathcode`q=\tlx@c \mathcode`r=\tlx@c \mathcode`s=\tlx@c \mathcode`t=\tlx@c
\mathcode`u=\tlx@c \mathcode`v=\tlx@c \mathcode`w=\tlx@c \mathcode`x=\tlx@c
\mathcode`y=\tlx@c \mathcode`z=\tlx@c
\c@tlx@ctr=\itfam \multiply\c@tlx@ctr"100\relax \advance\c@tlx@ctr "7041\relax
\mathcode`A=\tlx@c \mathcode`B=\tlx@c \mathcode`C=\tlx@c \mathcode`D=\tlx@c
\mathcode`E=\tlx@c \mathcode`F=\tlx@c \mathcode`G=\tlx@c \mathcode`H=\tlx@c
\mathcode`I=\tlx@c \mathcode`J=\tlx@c \mathcode`K=\tlx@c \mathcode`L=\tlx@c
\mathcode`M=\tlx@c \mathcode`N=\tlx@c \mathcode`O=\tlx@c \mathcode`P=\tlx@c
\mathcode`Q=\tlx@c \mathcode`R=\tlx@c \mathcode`S=\tlx@c \mathcode`T=\tlx@c
\mathcode`U=\tlx@c \mathcode`V=\tlx@c \mathcode`W=\tlx@c \mathcode`X=\tlx@c
\mathcode`Y=\tlx@c \mathcode`Z=\tlx@c
\makeatother

%%%%%%%%%%%%%%%%%%%%%%%%%%%%%%%%%%%%%%%%%%%%%%%%%%%%%%%%%%
%                THE describe ENVIRONMENT                %
%%%%%%%%%%%%%%%%%%%%%%%%%%%%%%%%%%%%%%%%%%%%%%%%%%%%%%%%%%
%
%
% It is like the description environment except it takes an argument
% ARG that should be the text of the widest label.  It adjusts the
% indentation so each item with label LABEL produces
%%      LABEL             blah blah blah
%%      <- width of ARG ->blah blah blah
%%                        blah blah blah
\newenvironment{describe}[1]%
   {\begin{list}{}{\settowidth{\labelwidth}{#1}%
            \setlength{\labelsep}{.5em}%
            \setlength{\leftmargin}{\labelwidth}% 
            \addtolength{\leftmargin}{\labelsep}%
            \addtolength{\leftmargin}{\parindent}%
            \def\makelabel##1{\rm ##1\hfill}}%
            \setlength{\topsep}{0pt}}%% 
                % Sets \topsep to 0 to reduce vertical space above
                % and below embedded displayed equations
   {\end{list}}

%   For tlatex.TeX
\usepackage{verbatim}
\makeatletter
\def\tla{\let\%\relax%
         \@bsphack
         \typeout{\%{\the\linewidth}}%
             \let\do\@makeother\dospecials\catcode`\^^M\active
             \let\verbatim@startline\relax
             \let\verbatim@addtoline\@gobble
             \let\verbatim@processline\relax
             \let\verbatim@finish\relax
             \verbatim@}
\let\endtla=\@esphack

\let\pcal=\tla
\let\endpcal=\endtla
\let\ppcal=\tla
\let\endppcal=\endtla

% The tlatex environment is used by TLATeX.TeX to typeset TLA+.
% TLATeX.TLA starts its files by writing a \tlatex command.  This
% command/environment sets \parindent to 0 and defines \% to its
% standard definition because the writing of the log files is messed up
% if \% is defined to be something else.  It also executes
% \@computerule to determine the dimensions for the TLA horizonatl
% bars.
\newenvironment{tlatex}{\@computerule%
                        \setlength{\parindent}{0pt}%
                       \makeatletter\chardef\%=`\%}{}


% The notla environment produces no output.  You can turn a 
% tla environment to a notla environment to prevent tlatex.TeX from
% re-formatting the environment.

\def\notla{\let\%\relax%
         \@bsphack
             \let\do\@makeother\dospecials\catcode`\^^M\active
             \let\verbatim@startline\relax
             \let\verbatim@addtoline\@gobble
             \let\verbatim@processline\relax
             \let\verbatim@finish\relax
             \verbatim@}
\let\endnotla=\@esphack

\let\nopcal=\notla
\let\endnopcal=\endnotla
\let\noppcal=\notla
\let\endnoppcal=\endnotla

%%%%%%%%%%%%%%%%%%%%%%%% end of tlatex.sty file %%%%%%%%%%%%%%%%%%%%%%% 
% last modified on Fri  3 August 2012 at 14:23:49 PST by lamport

\begin{document}
\tlatex
\@x{}\moduleLeftDash\@xx{ {\MODULE} gemini}\moduleRightDash\@xx{}%
\begin{lcom}{0}%
\begin{cpar}{0}{F}{F}{0}{0}{}%
 This is a specification for the \ensuremath{Gemini} Protocol. It describes
 the behavior
 of the submodules that make up \ensuremath{Gemini}\mbox{'}s two Top of Rack
 (\ensuremath{TOR}) solution for
 reliability. The purpose of the specification is \ensuremath{t} find all
 possible failure
 scenarios in the algorithm by exploring reachable states simulating
 transitions in the submodules\mbox{'} state machines.
\end{cpar}%
\end{lcom}%
\@x{ {\EXTENDS} FiniteSets}%
\@pvspace{8.0pt}%
\@x{ {\VARIABLES}}%
\@x{\@s{16.4} torA ,\,}%
\@x{\@s{16.4} torB ,\,}%
\@x{\@s{16.4} mux\@s{24.59}}%
\@y{\@s{0}%
 Which \ensuremath{ToR} the \ensuremath{MUX} cable itself is pointing to
}%
\@xx{}%
\begin{lcom}{16.4}%
\begin{cpar}{0}{F}{F}{0}{0}{}%
 \ensuremath{LinkProber} knows wether the \ensuremath{TOR} it\mbox{'}s hosted
 in should be active or standby
 by listening to the active \ensuremath{ToR}\mbox{'}s heartbeat that is sent
 to both \ensuremath{ToR}\mbox{'}s. It
 knows this because this \ensuremath{ICMP} ``heartbeat'' has the name or
 \ensuremath{MAC} address of the
 active \ensuremath{ToR}.
\end{cpar}%
\vshade{5.0}%
\begin{cpar}{0}{F}{F}{0}{0}{}%
 In the specification, this is represented by the \ensuremath{heartbeatSender}
 variable
 since the standby \ensuremath{ToR}\mbox{'}s heartbeat will be dropped and
 never listened to.
\end{cpar}%
\end{lcom}%
\@pvspace{8.0pt}%
\@x{ vars \.{\defeq} {\langle} torA ,\, torB ,\, mux {\rangle}}%
\@pvspace{8.0pt}%
\@x{ T \.{\defeq} \{\@w{torA} ,\,\@w{torB} \}}%
\@pvspace{8.0pt}%
\@x{}%
\@y{\@s{0}%
 Link \ensuremath{Prober} (page 9)
}%
\@xx{}%
 \@x{ LPStates \.{\defeq} \{\@w{LPActive} ,\,\@w{LPStandby} ,\,\@w{LPWait}
 ,\,\@w{LPUnknown} \}}%
\@pvspace{8.0pt}%
\@x{}%
\@y{\@s{0}%
 Link \ensuremath{Stat} (page 10)
}%
\@xx{}%
\@x{ LinkStates \.{\defeq} \{\@w{LinkUp} ,\,\@w{LinkDown} \}}%
\@pvspace{8.0pt}%
\@x{}%
\@y{\@s{0}%
 Mux State (page 12)
}%
\@xx{}%
 \@x{ MuxStates \.{\defeq} \{\@w{MuxActive} ,\,\@w{MuxStandby} ,\,\@w{MuxWait}
 ,\,\@w{MuxUnknown} \}}%
\@pvspace{8.0pt}%
\@x{}%
\@y{\@s{0}%
 \ensuremath{MUX\_XCVRD\_} (page 11)
}%
\@xx{}%
\@x{ XCVRDStates \.{\defeq} \{\@w{switch} ,\,\@w{check} ,\,\@w{{-}} \}}%
\@pvspace{8.0pt}%
\@x{ ToR \.{\defeq}}%
\@x{\@s{19.18} [ dead\@s{3.80} \.{:} {\BOOLEAN} ,\,}%
\@x{\@s{21.95} name \.{:} T ,\,}%
\@x{\@s{21.95} xcvrd \.{:} XCVRDStates ,\,}%
\@x{\@s{21.95} heartbeat \.{:} \{\@w{on} ,\,\@w{off} \} ,\,}%
 \@x{\@s{21.95} heartbeatIn \.{:} {\SUBSET} ( T \.{\cup} \{\@w{noResponse} \}
 ) ,\,}%
\@x{\@s{21.95} linkProber \.{:} LPStates ,\,}%
\@x{\@s{21.95} linkState \.{:} LinkStates ,\,}%
\@x{\@s{21.95} muxState \.{:} MuxStates ]}%
\@pvspace{8.0pt}%
\@x{}%
\@y{\@s{0}%
 ``Goal'' state for a \ensuremath{ToR}.
}%
\@xx{}%
\@x{ ActiveTor \.{\defeq}}%
\@x{\@s{16.4} [ dead\@s{3.80} \.{:} \{ {\FALSE} \} ,\,}%
\@x{\@s{19.17} name \.{:} T ,\,}%
\@x{\@s{19.17} xcvrd \.{:} \{\@w{{-}} \} ,\,}%
\@x{\@s{19.17} heartbeat \.{:} \{\@w{on} \} ,\,}%
 \@x{\@s{19.17} heartbeatIn \.{:} {\SUBSET} ( T \.{\cup} \{\@w{noResponse} \}
 ) ,\,}%
\@x{\@s{19.17} linkProber \.{:} \{\@w{LPActive} \} ,\,}%
\@x{\@s{19.17} linkState \.{:} \{\@w{LinkUp} \} ,\,}%
\@x{\@s{19.17} muxState \.{:} \{\@w{MuxActive} \} ]}%
\@pvspace{8.0pt}%
\@x{ TypeOK \.{\defeq}}%
\@x{\@s{16.4} \.{\land}\@s{9.74} torA\@s{0.63} \.{\in} ToR}%
\@x{\@s{16.4} \.{\land}\@s{9.74} torB \.{\in} ToR}%
 \@x{\@s{16.4} \.{\land}\@s{9.74} mux \.{\in} [ active \.{:} T ,\, next \.{:}
 T ]}%
\@pvspace{8.0pt}%
\@x{ Init \.{\defeq}}%
\@x{\@s{16.4} \.{\LET} InitialTor ( name ) \.{\defeq}}%
\@x{\@s{40.89} [ dead\@s{45.1} \.{\mapsto} {\FALSE} ,\,}%
\@x{\@s{43.67} name\@s{41.29} \.{\mapsto} name ,\,}%
\@x{\@s{43.67} xcvrd\@s{41.86} \.{\mapsto}\@w{check} ,\,}%
\@x{\@s{43.67} heartbeat\@s{26.14} \.{\mapsto}\@w{on} ,\,}%
\@x{\@s{43.67} heartbeatIn\@s{16.84} \.{\mapsto} \{ \} ,\,}%
\@x{\@s{43.67} linkProber\@s{20.5} \.{\mapsto}\@w{LPUnknown} ,\,}%
\@x{\@s{43.67} linkState\@s{26.98} \.{\mapsto}\@w{LinkDown} ,\,}%
\@x{\@s{43.67} muxState\@s{24.64} \.{\mapsto}\@w{MuxWait} ]}%
 \@x{\@s{16.4} \.{\IN}\@s{4.09} \.{\land} mux \.{\in} \{ f \.{\in} [ active
 \.{:} T ,\, next \.{:} T ] \.{:} f . active \.{=} f . next \}}%
\@x{\@s{40.89} \.{\land} torA\@s{0.63} \.{=} InitialTor (\@w{torA} )}%
\@x{\@s{40.89} \.{\land} torB \.{=} InitialTor (\@w{torB} )}%
\@pvspace{8.0pt}%
\@x{}\midbar\@xx{}%
\@x{}%
\@y{\@s{0}%
 State machine and transition table pages 12 \& 13 of the
 \ensuremath{Powerpoint} presentation as of 08/25/2022
}%
\@xx{}%
\@x{}%
\@y{\@s{0}%
 \ensuremath{XCVRD} daemon described on page 11 of the
 \ensuremath{Powerpoint} presentation as of 08/25/2022
}%
\@xx{}%
\@x{}%
\@y{\@s{0}%
 https:\ensuremath{\.{\slsl}}microsoft-\ensuremath{my.sharepoint.com}/:u:/p/\ensuremath{t}-gegranados/ERThXZdF5MVFusk2rP-PF0cBGguDR3Rt9yJ3WxxwAt0hpg\.{?}\ensuremath{e\.{=}i8aS4v
}}%
\@xx{}%
\@pvspace{8.0pt}%
\@x{}%
\@y{\@s{0}%
 Merged \ensuremath{LinkWait} and \ensuremath{MuxWait} on
 \ensuremath{Powerpoint} slide 13 into Wait
}%
\@xx{}%
\@pvspace{8.0pt}%
\begin{lcom}{0}%
\begin{cpar}{0}{F}{F}{0}{0}{}%
 \ensuremath{MuxState} State Transitions depend on
 \ensuremath{LinkManager}\mbox{'}s decisions and \ensuremath{XCVRD} responses
 when checking
 or switching the \ensuremath{MuxCable}\mbox{'}s direction
\end{cpar}%
\end{lcom}%
\@pvspace{8.0pt}%
\@x{ TRIGGER\_LINKMANAGER\_CHECK ( t ) \.{\defeq}}%
\begin{lcom}{16.4}%
\begin{cpar}{0}{F}{F}{0}{0}{}%
Sends check request to \ensuremath{MUX} via \ensuremath{XCVRD}.
 Transitions \ensuremath{muxState} to \ensuremath{MuxWait}.
 Defined to use in subsequent \ensuremath{MuxState} actions.
\end{cpar}%
\end{lcom}%
 \@x{\@s{16.4} t \.{'} \.{=} [ t {\EXCEPT} {\bang} . muxState
 \.{=}\@w{MuxWait} ,\, {\bang} . xcvrd \.{=}\@w{check} ]}%
\@pvspace{16.0pt}%
\@x{ TRIGGER\_LINKMANAGER\_SWITCH ( t ,\, target ) \.{\defeq}}%
\begin{lcom}{16.4}%
\begin{cpar}{0}{F}{F}{0}{0}{}%
Sends write request to \ensuremath{MUX} via \ensuremath{XCVRD}.
 Transitions \ensuremath{muxState} to \ensuremath{MuxWait} and
 \ensuremath{linkProber} to \ensuremath{LPWait
} target refers to \ensuremath{ToR} the \ensuremath{Mux} should point to.
 Defined to use in subsequent \ensuremath{MuxState} actions.
\end{cpar}%
\end{lcom}%
 \@x{\@s{16.4} \.{\land}\@s{4.1} t \.{'} \.{=} [ t {\EXCEPT} {\bang} .
 muxState\@s{8.45} \.{=}\@w{MuxWait} ,\, {\bang} . xcvrd \.{=}\@w{switch} ,\,
 {\bang} . linkProber \.{=}\@w{LPWait} ]}%
 \@x{\@s{16.4} \.{\land}\@s{4.09} mux \.{'} \.{=} [ mux {\EXCEPT} {\bang} .
 next \.{=} target . name ]}%
\@pvspace{8.0pt}%
\@x{ EXEC\_LINKMANAGER\_CHECK ( t ,\, otherTor ) \.{\defeq}}%
\@x{\@s{16.4} \.{\land} {\UNCHANGED} {\langle} mux ,\, otherTor {\rangle}}%
\@x{\@s{16.4} \.{\land} t . xcvrd \.{=}\@w{check}}%
 \@x{\@s{16.4} \.{\land}\@s{4.1} \.{\lor}\@s{4.1} \.{\land} mux . active \.{=}
 t . name}%
 \@x{\@s{46.82} \.{\land} t \.{'} \.{=} [ t {\EXCEPT} {\bang} . muxState
 \.{=}\@w{MuxActive} ,\, {\bang} . heartbeat \.{=}\@w{on} ,\, {\bang} . xcvrd
 \.{=}\@w{{-}} ]}%
\@x{\@s{31.61} \.{\lor}\@s{4.10} \.{\land} mux . active \.{\neq} t . name}%
 \@x{\@s{46.82} \.{\land} t \.{'} \.{=} [ t {\EXCEPT} {\bang} . muxState
 \.{=}\@w{MuxStandby} ,\, {\bang} . heartbeat\@s{5.01} \.{=}\@w{on} ,\,
 {\bang} . xcvrd \.{=}\@w{{-}} ]}%
 \@x{\@s{31.61} \.{\lor}\@s{4.10} \.{\land} t \.{'} \.{=} [ t {\EXCEPT}
 {\bang} . muxState \.{=}\@w{MuxUnknown} ,\, {\bang} . heartbeat \.{=}\@w{on}
 ,\, {\bang} . xcvrd \.{=}\@w{{-}} ]}%
\@x{\@s{31.61}}%
\@y{\@s{0}%
 \ensuremath{\.{\lor} \.{\land} t.muxState \.{=} \@w{MuxWait}
}}%
\@xx{}%
\@x{\@s{31.61}}%
\@y{\@s{10.0}%
 \ensuremath{\.{\land} t\.{'} \.{=}} [\ensuremath{t} \ensuremath{{\EXCEPT}
 {\bang}.muxState \.{=} \@w{MuxStandby}}, \ensuremath{{\bang}.xcvrd \.{=}
 \@w{-}}] \ensuremath{MUX\_XCVRD\_FAIL
}}%
\@xx{}%
\@pvspace{16.0pt}%
\@x{ EXEC\_LINKMANAGER\_SWITCH ( t ,\, otherTor ) \.{\defeq}}%
\@x{\@s{16.4}}%
\@y{\@s{0}%
 Writing Switch direction
}%
\@xx{}%
\@x{\@s{16.4} \.{\land} {\UNCHANGED} otherTor}%
\@x{\@s{16.4} \.{\land} t . xcvrd \.{=}\@w{switch}}%
\@x{\@s{16.4} \.{\land} t . muxState \.{=}\@w{MuxWait}}%
\@x{\@s{16.4} \.{\land} t . linkProber \.{=}\@w{LPWait}}%
 \@x{\@s{16.4} \.{\land} mux \.{'} \.{=} [ mux {\EXCEPT} {\bang} . active
 \.{=} mux . next ]}%
 \@x{\@s{16.4} \.{\land}\@s{4.1} \.{\lor}\@s{4.1} \.{\land} mux . next \.{=} t
 . name}%
 \@x{\@s{46.82} \.{\land} t \.{'} \.{=} [ t {\EXCEPT} {\bang} . muxState
 \.{=}\@w{MuxActive} ,\, {\bang} . heartbeat \.{=}\@w{on} ,\, {\bang} . xcvrd
 \.{=}\@w{{-}} ]}%
\@x{\@s{31.61} \.{\lor}\@s{4.10} \.{\land} mux . next \.{\neq} t . name}%
 \@x{\@s{46.82} \.{\land} t \.{'} \.{=} [ t {\EXCEPT} {\bang} . muxState
 \.{=}\@w{MuxStandby} ,\, {\bang} . heartbeat\@s{5.01} \.{=}\@w{on} ,\,
 {\bang} . xcvrd \.{=}\@w{{-}} ]}%
 \@x{\@s{31.61} \.{\lor}\@s{4.10} \.{\land} t \.{'} \.{=} [ t {\EXCEPT}
 {\bang} . muxState \.{=}\@w{MuxUnknown} ,\, {\bang} . heartbeat \.{=}\@w{on}
 ,\, {\bang} . xcvrd \.{=}\@w{{-}} ]}%
\@x{\@s{31.61}}%
\@y{\@s{0}%
 \ensuremath{\.{\lor} \.{\land} t.muxState \.{=} \@w{MuxWait}
}}%
\@xx{}%
\@x{\@s{31.61}}%
\@y{\@s{10.0}%
 \ensuremath{\.{\land} t\.{'} \.{=}} [\ensuremath{t} \ensuremath{{\EXCEPT}
 {\bang}.muxState \.{=} \@w{MuxStandby}}, \ensuremath{{\bang}.xcvrd \.{=}
 \@w{-}}] \ensuremath{MUX\_XCVRD\_FAIL
}}%
\@xx{}%
\@pvspace{8.0pt}%
\@x{}\midbar\@xx{}%
\@pvspace{8.0pt}%
\@x{ MuxStateActive ( t ,\, otherTor ) \.{\defeq}}%
\@x{\@s{16.4} \.{\land}\@s{4.1} t . muxState \.{=}\@w{MuxActive}}%
 \@x{\@s{16.4} \.{\land}\@s{4.09} \.{\lor}\@s{4.1} \.{\land} t . linkState
 \.{=}\@w{LinkUp}}%
\@x{\@s{46.82}}%
\@y{\@s{0}%
 \ensuremath{LinkUp} \ensuremath{MuxStateActive} Row
}%
\@xx{}%
 \@x{\@s{46.82} \.{\land}\@s{4.1} \.{\lor}\@s{4.1} \.{\land} t . linkProber
 \.{\in} \{\@w{LPStandby} ,\,\@w{LPUnknown} \}}%
\@x{\@s{77.24} \.{\land} TRIGGER\_LINKMANAGER\_CHECK ( t )}%
\@x{\@s{62.03} \.{\lor}\@s{4.10} \.{\land} t . linkProber \.{=}\@w{LPWait}}%
\@x{\@s{77.24}}%
\@y{\@s{0}%
 Check and suspend heartbeat
}%
\@xx{}%
 \@x{\@s{77.24} \.{\land} t \.{'} \.{=} [ t {\EXCEPT} {\bang} . muxState
 \.{=}\@w{MuxWait} ,\, {\bang} . xcvrd \.{=}\@w{check} ,\, {\bang} .
 heartbeat \.{=}\@w{off} ]}%
\@x{\@s{46.82} \.{\land} {\UNCHANGED} {\langle} mux ,\, otherTor {\rangle}}%
\@x{\@s{31.61} \.{\lor}\@s{4.10} \.{\land} t . linkState \.{=}\@w{LinkDown}}%
\@x{\@s{46.82}}%
\@y{\@s{0}%
 Switch to Standby
}%
\@xx{}%
\@x{\@s{46.82} \.{\land} TRIGGER\_LINKMANAGER\_SWITCH ( t ,\, otherTor )}%
\@x{\@s{46.82} \.{\land} {\UNCHANGED} otherTor}%
\@pvspace{8.0pt}%
\@x{ MuxStateStandby ( t ,\, otherTor ) \.{\defeq}}%
\@x{\@s{16.4} \.{\land}\@s{4.1} t . muxState \.{=}\@w{MuxStandby}}%
 \@x{\@s{16.4} \.{\land}\@s{4.09} \.{\lor}\@s{4.1} \.{\land} t . linkState
 \.{=}\@w{LinkUp}}%
\@x{\@s{31.61}}%
\@y{\@s{0}%
 \ensuremath{LinkUp} \ensuremath{MuxStateStandby} Row
}%
\@xx{}%
 \@x{\@s{48.01} \.{\land}\@s{4.1} \.{\lor}\@s{4.1} \.{\land} t . linkProber
 \.{\in} \{\@w{LPActive} ,\,\@w{LPWait} \}}%
\@x{\@s{78.43} \.{\land} TRIGGER\_LINKMANAGER\_CHECK ( t )}%
\@x{\@s{78.43} \.{\land} {\UNCHANGED} {\langle} mux ,\, otherTor {\rangle}}%
 \@x{\@s{63.22} \.{\lor}\@s{4.10} \.{\land} t . linkProber
 \.{=}\@w{LPUnknown}}%
\@x{\@s{78.43}}%
\@y{\@s{0}%
 Switch to Active
}%
\@xx{}%
\@x{\@s{78.43} \.{\land} TRIGGER\_LINKMANAGER\_SWITCH ( t ,\, t )}%
\@x{\@s{78.43} \.{\land} {\UNCHANGED} otherTor}%
\@x{\@s{31.61} \.{\lor}\@s{5.28} \.{\land} t . linkState \.{=}\@w{LinkDown}}%
\@x{\@s{31.61}}%
\@y{\@s{0}%
 \ensuremath{LinkDown} \ensuremath{MuxStateStandby} Row
}%
\@xx{}%
\@x{\@s{48.01} \.{\land} t . linkProber \.{\in} \{\@w{LPUnknown,\ LPWait} \}}%
\@x{\@s{48.01} \.{\land} TRIGGER\_LINKMANAGER\_CHECK ( t )}%
\@x{\@s{48.01} \.{\land} {\UNCHANGED} {\langle} mux ,\, otherTor {\rangle}}%
\@pvspace{8.0pt}%
\@x{ MuxStateUnknown ( t ,\, otherTor ) \.{\defeq}}%
\@x{\@s{16.4} \.{\land}\@s{4.1} t . muxState \.{=}\@w{MuxUnknown}}%
 \@x{\@s{16.4} \.{\land}\@s{4.09} \.{\lor}\@s{4.1} \.{\land} t . linkState
 \.{=}\@w{LinkUp}}%
\@x{\@s{46.82}}%
\@y{\@s{0}%
 \ensuremath{LinkUp} \ensuremath{MuxStateStandby} Row
}%
\@xx{}%
\@x{\@s{46.82} \.{\land} TRIGGER\_LINKMANAGER\_CHECK ( t )}%
\@x{\@s{31.61} \.{\lor}\@s{4.10} \.{\land} t . linkState \.{=}\@w{LinkDown}}%
\@x{\@s{31.61}}%
\@y{\@s{0}%
 \ensuremath{LinkDown} \ensuremath{MuxStateStandby} Row
}%
\@xx{}%
\@x{\@s{48.01} \.{\land} t . linkProber \.{\in} \{\@w{LPUnknown,\ LPWait} \}}%
\@x{\@s{48.01} \.{\land} TRIGGER\_LINKMANAGER\_CHECK ( t )}%
 \@x{\@s{16.4} \.{\land}\@s{4.10} {\UNCHANGED} {\langle} mux ,\, otherTor
 {\rangle}}%
\@pvspace{8.0pt}%
\@x{ MuxStateWait ( t ,\, otherTor ) \.{\defeq}}%
\@x{\@s{16.4}}%
\@y{%
 \ensuremath{TODO} Specify receiving \ensuremath{XCVRD} Response
}%
\@xx{}%
\@x{\@s{16.4} \.{\land} t . muxState \.{=}\@w{MuxWait}}%
 \@x{\@s{16.4} \.{\land}\@s{4.1} \.{\lor}\@s{4.1} \.{\land}
 EXEC\_LINKMANAGER\_CHECK ( t ,\, otherTor )}%
 \@x{\@s{31.61} \.{\lor}\@s{4.10} \.{\land} EXEC\_LINKMANAGER\_SWITCH ( t ,\,
 otherTor )}%
\@pvspace{8.0pt}%
\@x{ MuxState ( t ,\, otherTor ) \.{\defeq}}%
\@x{\@s{16.4} \.{\land} {\lnot} t . dead}%
 \@x{\@s{16.4} \.{\land}\@s{4.1} \.{\lor}\@s{4.1} MuxStateActive ( t ,\,
 otherTor )}%
\@x{\@s{31.61} \.{\lor}\@s{4.10} MuxStateStandby ( t ,\, otherTor )}%
\@x{\@s{31.61} \.{\lor}\@s{4.10} MuxStateUnknown ( t ,\, otherTor )}%
\@x{\@s{31.61} \.{\lor}\@s{4.10} MuxStateWait ( t ,\, otherTor )}%
\@pvspace{24.0pt}%
\@x{}\midbar\@xx{}%
\@pvspace{8.0pt}%
\@x{}%
\@y{\@s{0}%
 \ensuremath{MuxXCVRD \.{\defeq}
}}%
\@xx{}%
\@x{}%
\@y{\@s{10.0}%
 \ensuremath{\.{\,\backslash\,}}*
}%
\@xx{}%
\@x{}%
\@y{\@s{10.0}%
 \ensuremath{\.{\land} {\UNCHANGED} {\langle}torA,\, torB{\rangle}
}}%
\@xx{}%
\@x{}%
\@y{\@s{10.0}%
 \ensuremath{\.{\land} mux\.{'} \.{=}} [ \ensuremath{mux}
 \ensuremath{{\EXCEPT} {\bang}.active \.{=} mux.next} ]
}%
\@xx{}%
\@pvspace{8.0pt}%
\@x{}%
\@y{\@s{0}%
 State machine page 10 of the \ensuremath{Powerpoint} presentation as of
 08/25/2022
}%
\@xx{}%
\@pvspace{8.0pt}%
\@x{ LinkState ( t ,\, otherTor ) \.{\defeq}}%
\@x{\@s{16.4} \.{\land} {\lnot} t . dead}%
\@x{\@s{16.4} \.{\land} t . linkState \.{=}\@w{LinkDown}}%
\@y{\@s{0}%
 unnecessary, because going from \ensuremath{LinkUp} to \ensuremath{LinkUp}
 is just (finite) stuttering. However, this conjunct prevent the debugger
 from evaluating this action when it is stuttering.
}%
\@xx{}%
\@x{\@s{16.4} \.{\land} {\UNCHANGED} {\langle} otherTor ,\, mux {\rangle}}%
 \@x{\@s{16.4} \.{\land} t \.{'} \.{=} [ t {\EXCEPT} {\bang} . linkState
 \.{=}\@w{LinkUp} ]}%
\@pvspace{8.0pt}%
\@x{}\midbar\@xx{}%
\@pvspace{8.0pt}%
\@x{}%
\@y{\@s{0}%
 State machine page 09 of the \ensuremath{Powerpoint} presentation as of
 08/25/2022
}%
\@xx{}%
\@x{}%
\@y{\@s{0}%
 https:\ensuremath{\.{\slsl}}microsoft-\ensuremath{my.sharepoint.com}/:u:/p/zhangjing/\ensuremath{EclAzBSCq\_5KuwgbbpyUlMQB1RS\_X9nibOrM1PjT8wM\_uw}\.{?}\ensuremath{e\.{=}eBtJKl
}}%
\@xx{}%
\@pvspace{8.0pt}%
\@x{ SendHeartbeat ( t ) \.{\defeq}}%
\@x{\@s{16.4} \.{\land} {\lnot} t . dead}%
\@x{\@s{16.4} \.{\land} t . linkState\@s{0.83} \.{=}\@w{LinkUp}}%
\@x{\@s{16.4} \.{\land} t . heartbeat \.{=}\@w{on}}%
\begin{lcom}{16.4}%
\begin{cpar}{0}{F}{F}{0}{0}{}%
 Active \ensuremath{ToR} sends heartbeat to server. \ensuremath{MUX}
 duplicates packet and sends it
 to both \ensuremath{ToR}\mbox{'}s
\end{cpar}%
\end{lcom}%
\@x{\@s{16.4} \.{\land}\@s{4.1} mux . active \.{=} t . name\@s{4.1}}%
\@y{\@s{0}%
 The \ensuremath{MUX} will drop traffic from \ensuremath{ToR} if it is not
 pointing to it
}%
\@xx{}%
 \@x{\@s{16.4} \.{\land} torA \.{'}\@s{0.63} \.{=} [ torA {\EXCEPT} {\bang} .
 heartbeatIn\@s{0.63} \.{=} @ \.{\cup} \{ t . name \} ]}%
 \@x{\@s{16.4} \.{\land} torB \.{'} \.{=} [ torB {\EXCEPT} {\bang} .
 heartbeatIn \.{=} @ \.{\cup} \{ t . name \} ]}%
\@x{\@s{16.4} \.{\land} {\UNCHANGED} {\langle} mux {\rangle}}%
\@pvspace{8.0pt}%
\@x{ LinkProberWait ( t ,\, otherTor ) \.{\defeq}}%
\@x{\@s{16.4} \.{\land} {\UNCHANGED} {\langle} otherTor ,\, mux {\rangle}}%
\@x{\@s{16.4} \.{\land} t . linkState \.{=}\@w{LinkUp}}%
\@x{\@s{16.4} \.{\land} t . linkProber \.{=}\@w{LPWait}}%
 \@x{\@s{16.4} \.{\land} \E\, heartbeat\@s{5.84} \.{\in} t . heartbeatIn
 \.{:}}%
\@x{\@s{31.61} \.{\lor} \.{\land} t . name \.{=} heartbeat}%
 \@x{\@s{42.72} \.{\land} t \.{'} \.{=} [ t {\EXCEPT} {\bang} . linkProber
 \.{=}\@w{LPActive} ,\, {\bang} . heartbeatIn \.{=} @ \.{\,\backslash\,} \{
 heartbeat \} ]}%
\@x{\@s{31.61} \.{\lor} \.{\land} otherTor . name \.{=} heartbeat}%
 \@x{\@s{42.72} \.{\land} t \.{'} \.{=} [ t {\EXCEPT} {\bang} . linkProber
 \.{=}\@w{LPStandby} ,\, {\bang} . heartbeatIn \.{=} @ \.{\,\backslash\,} \{
 heartbeat \} ]}%
\@x{\@s{31.61} \.{\lor} \.{\land}\@w{noResponse} \.{=} heartbeat}%
 \@x{\@s{42.72} \.{\land} t \.{'} \.{=} [ t {\EXCEPT} {\bang} . heartbeatIn
 \.{=} @ \.{\,\backslash\,} \{ heartbeat \} ]}%
\@pvspace{16.0pt}%
\@x{ LinkProberUnknown ( t ,\, otherTor ) \.{\defeq}}%
\@x{\@s{16.4} \.{\land} {\UNCHANGED} {\langle} otherTor ,\, mux {\rangle}}%
\@x{\@s{16.4} \.{\land} t . linkState \.{=}\@w{LinkUp}}%
\@x{\@s{16.4} \.{\land} t . linkProber \.{=}\@w{LPUnknown}}%
 \@x{\@s{16.4} \.{\land} \E\, heartbeat\@s{5.84} \.{\in} t . heartbeatIn
 \.{:}}%
\@x{\@s{31.61} \.{\lor} \.{\land} t . name \.{=} heartbeat}%
 \@x{\@s{42.72} \.{\land} t \.{'} \.{=} [ t {\EXCEPT} {\bang} . linkProber
 \.{=}\@w{LPActive} ,\, {\bang} . heartbeatIn \.{=} @ \.{\,\backslash\,} \{
 heartbeat \} ]}%
\@x{\@s{31.61} \.{\lor} \.{\land} otherTor . name \.{=} heartbeat}%
 \@x{\@s{42.72} \.{\land} t \.{'} \.{=} [ t {\EXCEPT} {\bang} . linkProber
 \.{=}\@w{LPStandby} ,\, {\bang} . heartbeatIn \.{=} @ \.{\,\backslash\,} \{
 heartbeat \} ]}%
\@x{\@s{31.61} \.{\lor} \.{\land}\@w{noResponse} \.{=} heartbeat}%
 \@x{\@s{42.72} \.{\land} t \.{'} \.{=} [ t {\EXCEPT} {\bang} . heartbeatIn
 \.{=} @ \.{\,\backslash\,} \{ heartbeat \} ]}%
\@pvspace{8.0pt}%
\@x{ LinkProberStandby ( t ,\, otherTor ) \.{\defeq}}%
\@x{\@s{16.4} \.{\land} {\UNCHANGED} {\langle} otherTor ,\, mux {\rangle}}%
\@x{\@s{16.4} \.{\land} t . linkState \.{=}\@w{LinkUp}}%
\@x{\@s{16.4} \.{\land} t . linkProber\@s{1.05} \.{=}\@w{LPStandby}}%
 \@x{\@s{16.4} \.{\land} \E\, heartbeat\@s{6.90} \.{\in} t . heartbeatIn
 \.{:}}%
\@x{\@s{27.51} \.{\lor} \.{\land} t . name \.{=} heartbeat}%
 \@x{\@s{38.62} \.{\land} t \.{'} \.{=} [ t {\EXCEPT} {\bang} . linkProber
 \.{=}\@w{LPActive} ,\, {\bang} . heartbeatIn \.{=} @ \.{\,\backslash\,} \{
 heartbeat \} ]}%
\@x{\@s{27.51} \.{\lor} \.{\land} otherTor . name \.{=} heartbeat}%
 \@x{\@s{38.62} \.{\land} t \.{'} \.{=} [ t {\EXCEPT} {\bang} . heartbeatIn
 \.{=} @ \.{\,\backslash\,} \{ heartbeat \} ]}%
\@x{\@s{27.51} \.{\lor} \.{\land}\@w{noResponse} \.{=} heartbeat}%
 \@x{\@s{38.62} \.{\land} t \.{'} \.{=} [ t {\EXCEPT} {\bang} . linkProber
 \.{=}\@w{LPUnknown} ,\, {\bang} . heartbeatIn \.{=} @ \.{\,\backslash\,} \{
 heartbeat \} ]}%
\@pvspace{8.0pt}%
\@x{ LinkProberActive ( t ,\, otherTor ) \.{\defeq}}%
\@x{\@s{16.4} \.{\land} {\UNCHANGED} {\langle} otherTor ,\, mux {\rangle}}%
\@x{\@s{16.4} \.{\land} t . linkState \.{=}\@w{LinkUp}}%
\@x{\@s{16.4} \.{\land} t . linkProber\@s{1.05} \.{=}\@w{LPActive}}%
 \@x{\@s{16.4} \.{\land} \E\, heartbeat\@s{6.90} \.{\in} t . heartbeatIn
 \.{:}}%
\@x{\@s{27.51} \.{\lor} \.{\land} t . name \.{=} heartbeat}%
 \@x{\@s{38.62} \.{\land} t \.{'} \.{=} [ t {\EXCEPT} {\bang} . heartbeatIn
 \.{=} @ \.{\,\backslash\,} \{ heartbeat \} ]}%
\@x{\@s{27.51} \.{\lor} \.{\land} otherTor . name \.{=} heartbeat}%
 \@x{\@s{38.62} \.{\land} t \.{'} \.{=} [ t {\EXCEPT} {\bang} . linkProber
 \.{=}\@w{LPStandby} ,\, {\bang} . heartbeatIn \.{=} @ \.{\,\backslash\,} \{
 heartbeat \} ]}%
\@x{\@s{27.51} \.{\lor} \.{\land}\@w{noResponse} \.{=} heartbeat}%
 \@x{\@s{38.62} \.{\land} t \.{'} \.{=} [ t {\EXCEPT} {\bang} . linkProber
 \.{=}\@w{LPUnknown} ,\, {\bang} . heartbeatIn \.{=} @ \.{\,\backslash\,} \{
 heartbeat \} ]}%
\@pvspace{8.0pt}%
\@x{ LinkProber ( t ,\, otherTor ) \.{\defeq}}%
\@x{\@s{16.4} \.{\land} {\lnot} t . dead}%
\@x{\@s{16.4} \.{\land}\@s{4.1} \.{\lor} LinkProberActive ( t ,\, otherTor )}%
\@x{\@s{31.61} \.{\lor} LinkProberStandby ( t ,\, otherTor )}%
\@x{\@s{31.61} \.{\lor} LinkProberWait ( t ,\, otherTor )}%
\@x{\@s{31.61} \.{\lor} LinkProberUnknown ( t ,\, otherTor )}%
\@pvspace{8.0pt}%
\@x{}\midbar\@xx{}%
\@pvspace{8.0pt}%
\@x{ System \.{\defeq}}%
\begin{lcom}{16.4}%
\begin{cpar}{0}{F}{F}{0}{0}{}%
Mux handling a switch or check command.
\end{cpar}%
\end{lcom}%
\@x{\@s{16.4} \.{\lor} EXEC\_LINKMANAGER\_SWITCH ( torA ,\, torB )}%
\@x{\@s{16.4} \.{\lor} EXEC\_LINKMANAGER\_SWITCH ( torB ,\, torA )}%
\@x{\@s{16.4} \.{\lor} EXEC\_LINKMANAGER\_CHECK ( torA ,\, torB )}%
\@x{\@s{16.4} \.{\lor} EXEC\_LINKMANAGER\_CHECK ( torB ,\, torA )}%
\begin{lcom}{16.4}%
\begin{cpar}{0}{F}{F}{0}{0}{}%
\ensuremath{XCVRD} and \ensuremath{LinkMgrd
}%
\end{cpar}%
\end{lcom}%
\@x{\@s{16.4} \.{\lor} MuxState ( torA ,\, torB )}%
\@x{\@s{16.4} \.{\lor} MuxState ( torB ,\, torA )}%
\begin{lcom}{16.4}%
\begin{cpar}{0}{F}{F}{0}{0}{}%
 \ensuremath{ToR} periodically send heartbeats via the \ensuremath{mux} to the
 server.
\end{cpar}%
\end{lcom}%
\@x{\@s{16.4} \.{\lor} SendHeartbeat ( torA )}%
\@x{\@s{16.4} \.{\lor} SendHeartbeat ( torB )}%
\begin{lcom}{16.4}%
\begin{cpar}{0}{F}{F}{0}{0}{}%
 \ensuremath{ToR} receives heartbeat and triggers appropriate transition in
 \ensuremath{LinkProber
}%
\end{cpar}%
\end{lcom}%
\@x{\@s{16.4} \.{\lor} LinkProber ( torA ,\, torB )}%
\@x{\@s{16.4} \.{\lor} LinkProber ( torB ,\, torA )}%
\begin{lcom}{16.4}%
\begin{cpar}{0}{F}{F}{0}{0}{}%
Notification from the kernel that a physical link (\ensuremath{L1}) came up.
\end{cpar}%
\end{lcom}%
\@x{\@s{16.4} \.{\lor} LinkState ( torA ,\, torB )}%
\@x{\@s{16.4} \.{\lor} LinkState ( torB ,\, torA )}%
\@pvspace{8.0pt}%
\@x{}\midbar\@xx{}%
\@pvspace{8.0pt}%
\@x{ FailHeartbeat \.{\defeq}}%
\begin{lcom}{16.4}%
\begin{cpar}{0}{F}{F}{0}{0}{}%
 Sender fails to send heartbeat to \ensuremath{ToR}\mbox{'}s making them go
 into unknown state
\end{cpar}%
\end{lcom}%
 \@x{\@s{16.4} \.{\land} \.{\lor} \.{\land} \E\, heartbeat \.{\in} {\SUBSET}
 torA . heartbeatIn \.{:}}%
 \@x{\@s{56.95} \.{\land} torA \.{'} \.{=} [ torA {\EXCEPT} {\bang} .
 heartbeatIn \.{=} heartbeat ]}%
\@x{\@s{56.95} \.{\land} {\UNCHANGED} torB}%
 \@x{\@s{27.51} \.{\lor} \.{\land} \E\, heartbeat \.{\in} {\SUBSET} torB .
 heartbeatIn \.{:}}%
 \@x{\@s{56.95} \.{\land} torB \.{'} \.{=} [ torB {\EXCEPT} {\bang} .
 heartbeatIn \.{=} heartbeat ]}%
\@x{\@s{56.95} \.{\land} {\UNCHANGED} torA}%
\@x{\@s{16.4} \.{\land} {\UNCHANGED} mux}%
\@pvspace{8.0pt}%
\@x{ FailMux \.{\defeq}}%
\begin{lcom}{16.4}%
\begin{cpar}{0}{F}{F}{0}{0}{}%
 Failure Action for inconsistent \ensuremath{MUX} States with
 \ensuremath{MuxCable} State
\end{cpar}%
\end{lcom}%
 \@x{\@s{16.4} \.{\land}\@s{4.1} mux \.{'} \.{\in} [ active \.{:} T ,\, next
 \.{:} T ]}%
 \@x{\@s{16.4} \.{\land}\@s{4.09} {\UNCHANGED} {\langle} torA ,\, torB
 {\rangle}}%
\@pvspace{8.0pt}%
\@x{ FailTor ( t ,\, otherTor ) \.{\defeq}}%
 \@x{\@s{16.4} \.{\land}\@s{5.77} t \.{'} \.{=} [ t {\EXCEPT} {\bang} . dead
 \.{=} {\TRUE} ]}%
 \@x{\@s{16.4} \.{\land}\@s{5.77} {\UNCHANGED} {\langle} otherTor ,\, mux
 {\rangle}}%
\@pvspace{8.0pt}%
\@x{ FailXCVRD ( t ,\, otherTor ) \.{\defeq}}%
\@x{\@s{16.4} \.{\land} {\UNCHANGED} {\langle} otherTor ,\, mux {\rangle}}%
\@x{\@s{16.4}}%
\@y{\@s{0}%
 According to \ensuremath{Vaibhav} \ensuremath{Dahiya}, the \ensuremath{mux}
 returns ``Unknown'' in case of failure,
}%
\@xx{}%
\@x{\@s{16.4}}%
\@y{\@s{0}%
 which, subsequently, causes the \ensuremath{ToR} to go to ``Standby''.
}%
\@xx{}%
 \@x{\@s{16.4} \.{\land} t \.{'} \.{=} [ t {\EXCEPT} {\bang} . xcvrd
 \.{=}\@w{{-}} ]}%
\@pvspace{8.0pt}%
\@x{ FailLinkState ( t ,\, otherTor ) \.{\defeq}}%
\@x{\@s{16.4} \.{\land} {\lnot} t . dead}%
\@x{\@s{16.4} \.{\land} {\UNCHANGED} {\langle} otherTor ,\, mux {\rangle}}%
 \@x{\@s{16.4} \.{\land} t \.{'} \.{=} [ t {\EXCEPT} {\bang} . linkState
 \.{=}\@w{LinkDown} ]}%
\@pvspace{8.0pt}%
\@x{ Environment \.{\defeq}}%
\@x{\@s{16.4} \.{\lor} FailMux}%
\@x{\@s{16.4} \.{\lor} FailHeartbeat}%
\@x{\@s{16.4} \.{\lor} FailTor ( torA ,\, torB )}%
\@x{\@s{16.4} \.{\lor} FailTor ( torB ,\, torA )}%
\@x{\@s{16.4} \.{\lor} FailXCVRD ( torA ,\, torB )}%
\@x{\@s{16.4} \.{\lor} FailXCVRD ( torB ,\, torA )}%
\@x{\@s{16.4} \.{\lor} FailLinkState ( torA ,\, torB )}%
\@x{\@s{16.4} \.{\lor} FailLinkState ( torB ,\, torA )}%
\@pvspace{8.0pt}%
\@x{}\midbar\@xx{}%
\@pvspace{8.0pt}%
\@x{ Next \.{\defeq}}%
\@x{\@s{16.4}}%
\@y{\@s{0}%
 \ensuremath{\.{\lor} Environment
}}%
\@xx{}%
\@x{\@s{16.4} \.{\lor} System}%
\@pvspace{8.0pt}%
\@x{ Spec \.{\defeq}}%
 \@x{\@s{16.4} Init \.{\land} {\Box} [ Next ]_{ vars} \.{\land} {\WF}_{ vars}
 ( System )}%
\@pvspace{8.0pt}%
\@x{}\midbar\@xx{}%
\@pvspace{8.0pt}%
\@x{}%
\@y{\@s{0}%
 \ensuremath{AtMostOneActive \.{\defeq}
}}%
\@xx{}%
\@x{}%
\@y{\@s{10.0}%
 \ensuremath{{\Box}{\lnot}(torA \.{\in} ActiveTor \.{\land} torB \.{\in}
 ActiveTor)
}}%
\@xx{}%
\@pvspace{8.0pt}%
\@x{ RepeatedlyOneActive \.{\defeq}}%
 \@x{\@s{16.4} {\Box} {\Diamond} ( \E\, t \.{\in} \{ torA ,\, torB \} \.{:}
 {\lnot} t . dead \.{\implies} ( torA \.{\in} ActiveTor \.{\lor} torB \.{\in}
 ActiveTor ) )}%
\@pvspace{8.0pt}%
\@x{ {\THEOREM} Spec \.{\implies}}%
\@x{\@s{16.4}}%
\@y{%
 \ensuremath{\.{\land} AtMostOneActive
}}%
\@xx{}%
\@x{\@s{16.4} \.{\land} RepeatedlyOneActive}%
\@pvspace{8.0pt}%
\@x{}\midbar\@xx{}%
\@pvspace{8.0pt}%
\@x{ Alias \.{\defeq}}%
\@x{\@s{16.4} [}%
 \@x{\@s{32.8} torA \.{\mapsto} torA ,\, torB \.{\mapsto} torB ,\, mux
 \.{\mapsto} mux ,\,}%
 \@x{\@s{32.8} active \.{\mapsto} \{ t . name \.{:} t \.{\in} \{ t \.{\in} \{
 torA ,\, torB \} \.{:} t \.{\in} ActiveTor \} \}}%
\@x{\@s{16.4} ]}%
\@x{}\bottombar\@xx{}%
\end{document}
